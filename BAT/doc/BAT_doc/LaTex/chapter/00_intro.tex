\chapter{Introduction}
This document will include the BAT (Bolt Analysis Tool) User Manual \cite{ESAPSS} \cite{ECSS_HB_32_23A} \cite{VDI2230_1}.
\begin{equation}
  p(\bm{\Theta}|\bm{y}) = \frac{p(\bm{y}|\bm{\Theta})\ p(\bm{\Theta})}{p(\bm{y})}\ ,
\end{equation}

\begin{figure}[!htpb]
  \centering
  \includegraphics[width=0.7\textwidth]{VDI2230_Bild24.png}
  \caption{Joint diagram for the working state of a concentrically loaded \\ bolted joint with $n=1$ \cite{VDI2230_1}}
  \label{fig:joint_diagram}
\end{figure}

\chapter{Bolt and Thread Geometry}
$D_{Km}$ is the \emph{effective diameter of under head/nut friction torque} and is defined by 
\begin{equation}
  D_{Km} = \frac{D_{hole}+d_h}{2}
  \label{equ:dkm}
\end{equation}
where $D_{hole}$ is the \emph{through-hole diameter} in the clamped parts and $d_h$ is the 
\emph{minimum bearing surface outer diamter} of the bolt head or nut.
\begin{figure}[!htpb]
  \centering
  \includegraphics[width=0.5\textwidth]{ECSS_uh_brg_angle.png}
  \caption{Definition of under head bearing angle \cite{ECSS_HB_32_23A}}
  \label{fig:ecss_uh_brg_angle}
\end{figure}

\chapter{Method B: ECSS-E-HB-32-23A}
This chapter provides a quick overview and summary of the equations used in \bat. A detailed description
can be found in the complete ECSS-E-HB-32-23A ESA handbook \cite{ECSS_HB_32_23A}. Some used variables
in the following equations have been changed compared to \cite{ECSS_HB_32_23A} by the author to increase
clarity and consistency.
\section{Preload and Torques}
The torque present at the thread interface $M_{th}$ is dependent of the \emph{axial bolt preload} $F_V$ 
and is given by
\begin{equation}
  M_{th} = F_V \tan(\varphi+\rho)\frac{d_2}{2}
\end{equation}
and the \emph{under-head torque} $M_{uh}$ due to friction between bolt head or nut and the adjacent 
clamped part (or shim) is defined by
\begin{equation}
  M_{uh} = F_V \frac{\mu_{uh} D_{Km}}{2} \frac{1}{\sin{\nicefrac{\lambda}{2}}}
\end{equation}
where $\lambda$ is the \emph{under head bearing angle} seen in \fig{fig:ecss_uh_brg_angle}.
It is assumed that the friction force for $M_{uh}$ is acting at mean bearing radius of the bolt head 
$D_{Km}$ \equ{equ:dkm}. $\varphi$ is the helix angle of the thread and $\rho$ is given by the relation
\begin{equation}
  \tan\rho = \frac{\mu_{th}}{\cos\nicefrac{\theta}{2}}
\end{equation}
where $\theta$ is the half angle of the thread groves (for Unified or Metric threads $\theta=60^\circ$).

The \emph{total installation torque} $T_A$ (without torque device scatter) applied to bolt head or nut
during tightening to produce the axial bolt preload $F_V$ is 
\begin{equation}
  T_A = M_{th} + M_{uh} + M_p
  \label{equ:TA}
\end{equation}
where $M_p$ is the \emph{prevailing torque} of the locking device. 
With the approximation $\tan\varphi \tan\rho \ll 1$ the expression $\tan(\varphi+\rho)$ can be written
as $\tan(\varphi+\rho) \approx \tan\varphi + \tan\rho$. Now equation \equ{equ:TA} can be rewritten to
\begin{equation}
  T_A = F_V \underbrace{ \left[ \frac{d_2}{2} \left( \tan\varphi + \frac{\mu_{th}}{\cos\nicefrac{\theta}{2}} \right) 
  + \frac{\mu_{uh} D_{Km}}{2 \sin\nicefrac{\lambda}{2}} \right]}_{K} + M_p
  \label{equ:TA2}
\end{equation}
where $K$ is the \emph{joint coefficient}.

For calculation of the minimum and maximum axial bolt preload, BAT implements the \emph{experimental coefficient method}
\cite{ECSS_HB_32_23A} with an explicit torque scatter torque of the tightening device $T_{scatter}$. Therefore
the minimum and maximum total installation torques are defined 
\begin{equation}
  T_A^{min} = T_A - T_{scatter} , \qquad T_A^{max} = T_A + T_{scatter}.
  \label{equ:Tscatter}
\end{equation}
To calculate the minimum and maximum \emph{axial bolt preload after tightening} $F_M^{min/max}$, 
\equ{equ:TA2} and \equ{equ:Tscatter} are combined 
\begin{equation}
  F_M^{min} = \frac{T_A^{min}-M_p^{max}}{K^{max}} ,\qquad
  F_M^{max} = \frac{T_A^{max}-M_p^{min}}{K^{min}}.
\end{equation}
If also the thermal influence and embedding is considered, this leads to the minimum and maximum
\emph{axial bolt preload at service} $F_V^{min/max}$
\begin{subequations}
  \begin{align}
    F_V^{min} &= \frac{T_A^{min}-M_p^{max}}{K^{max}}+\Delta F_{Vth}-F_Z \\
    F_V^{min} &= F_M^{min}+\Delta F_{Vth}-F_Z \\
    &= \frac{T_A^{min}-M_p^{max}}{\frac{d_2}{2} \left( \tan\varphi + \frac{\mu_{th}^{max}}
    {\cos\nicefrac{\theta}{2}} \right) + \frac{\mu_{uh}^{max} D_{Km}}
    {2 \sin\nicefrac{\lambda}{2}}}+\Delta F_{Vth}-F_Z
    \label{equ:FVmin}
  \end{align}
\end{subequations}
\begin{subequations}
  \begin{align}
    F_V^{max} &= \frac{T_A^{max}-M_p^{min}}{K^{min}}+\Delta F_{Vth} \\
    F_V^{max} &= F_M^{max}+\Delta F_{Vth} \\
    &= \frac{T_A^{max}-M_p^{min}}{\frac{d_2}{2} \left( \tan\varphi + \frac{\mu_{th}^{min}}
    {\cos\nicefrac{\theta}{2}} \right) + \frac{\mu_{uh}^{min} D_{Km}}
    {2 \sin\nicefrac{\lambda}{2}}}+\Delta F_{Vth}
    \label{equ:FVmax}
  \end{align}
\end{subequations}
where $\Delta F_{Vth}$ is thermal preload change and $F_Z$ is the preload loss due to embedding.